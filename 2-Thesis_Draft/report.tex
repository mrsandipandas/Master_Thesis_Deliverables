% twoside library is for both side control of document (in case you need blank paper)
%\documentclass[12pt,a4paper,twoside,openright,hidelinks, abstractoff,titlepage,final,parskip=half*,BCOR10mm]{scrreprt}

%Used scrreprt in place of report to achieve customized "chapter"
\documentclass[12pt,a4paper,openright,hidelinks,titlepage,final]{scrreprt}

% Cover page library
\usepackage[utf8]{inputenc}
\usepackage[eng,exjobb]{Ch0.1-Cover/KTHEEtitlepage}

% Common libraries
\usepackage{amsmath,booktabs,caption,color,graphicx,float,listings,lmodern,microtype,multirow,pifont,rotating,siunitx,subcaption,subfiles,tabularx,xcolor}
\graphicspath{{images/}{../images/}}

% For "Acronym" and "Glossary"  
\usepackage[acronym,toc,xindy]{glossaries}
\makeglossaries
%nonumberlist %do not show page numbers
%acronym      %generate acronym listing
%toc          %show listings as entries in table of contents
%section      %use section level for toc entries

% Import the glossary and acronym here

\usepackage[automark,komastyle,headsepline]{scrpage2}
\usepackage[pdfpagelabels]{hyperref}
\usepackage{amssymb}

%Includes "References" in the table of contents, add "numbib" to add "Chapter" in Bibliography section
\usepackage[nottoc,notlot,notlof]{tocbibind}
\usepackage[english]{babel}
\usepackage[square,numbers]{natbib}
\usepackage{url}

%%%%%%%%%%%%%%%%%%%%%%%Customized Chapter Title%%%%%%%%%%%%%%%%%%%%
\makeatletter
\renewcommand*{\chapterformat}{
	\begingroup
	\setlength{\unitlength}{1mm}%
		\begin{picture}(20,30)(0,5)%
		
		%----  cmd \line(1,0) is use to draw straight horizontalline----
		\put(20,15){\line(1,0){\dimexpr
			\textwidth-20\unitlength\relax\@gobble}}
		\put(0,0){\makebox(20,20)[r]{
			\fontsize{28\unitlength}{28\unitlength}\selectfont\thechapter
			\kern-.04em}}
			
		%----- for additional chapter label showup above the line
		\put(20,15){\makebox(\dimexpr
			\textwidth-20\unitlength\relax\@gobble,\ht\strutbox\@gobble)[l]{
			\normalsize\color{black}\chapapp~\thechapter\autodot}}
		\end{picture} 
	\endgroup
}
%%%%%%%%%%%%%%%%%%%%%%%%%%%%%%%%%%%%%%%%%%%%%%%%%%%%%%%%%%%%%%%%%%%

\begin{document}
%%%%%%%%%%%%%%%%%%%%%%%%%%%Foreword%%%%%%%%%%%%%%%%%%%%%%%%%%%%%%%%
    % These introductory pages have a roman numbering
    \pagenumbering{roman}
    % note: If folder path is changed you also have to change in 2 places: usepackage in main.tex and .sty file in Ch0.1-Cover
    \subfile{Ch0.1-Cover/KTHEEtitlepage_ex_exjobb}
    \subfile{Ch0.2-Title/title}
    %\subfile{Ch0.3-Acknowledgement/ack}
    %\subfile{Ch0.4-Abstract/abstract}      
    
    % Set counter so that table of content page numbering updated correctly
    %\setcounter{page}{3}
    \setcounter{tocdepth}{3}% Include \subsubsection in ToC
    \setcounter{secnumdepth}{3}% Number \subsubsection
    
    \glsunsetall %Unset glossary before to view acronym in TOC
    
    \tableofcontents
    
    \glsresetall %Reset glossary after to view normally later
    
    \newpage
%%%%%%%%%%%%%%%%%%%%%%%%%%%%%%%%%%%%%%%%%%%%%%%%%%%%%%%%%%%%%%%%%%%    

%%%%%%%%%%%%%%%%%%%%%%%%%%%Main Content%%%%%%%%%%%%%%%%%%%%%%%%%%%%    
    % These introductory pages have a arabic numeral numbering
    \pagenumbering{arabic}
    \subfile{Ch1-Introduction/ch1}
    \subfile{Ch2-State_of_the_art/ch2}
    \subfile{Ch3-Design_and_simulation/ch3}
    \subfile{Ch4-Fabrication/ch4}
    \subfile{Ch5-Results/ch5}
    \subfile{Ch6-Conclusion/ch6}
    \subfile{Ch7-Limitations_and_future_work/ch7}
    \subfile{Ch8-Appendix/glossary}
%%%%%%%%%%%%%%%%%%%%%%%%%%%%%%%%%%%%%%%%%%%%%%%%%%%%%%%%%%%%%%%%%%%

%%%%%%%%%%%%%%%%%%%%%%%%%%%Appendix%%%%%%%%%%%%%%%%%%%%%%%%%%%%%%%%    
    % Abbreviations
    \printglossary[type=acronym,title={Abbreviations}]   
    
    % Bibliography
    \bibliographystyle{ieeetr}
    \bibliography{Ch9-Bibliography/ref}   
%%%%%%%%%%%%%%%%%%%%%%%%%%%%%%%%%%%%%%%%%%%%%%%%%%%%%%%%%%%%%%%%%%%
\end{document}

