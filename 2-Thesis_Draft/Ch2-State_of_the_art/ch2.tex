\documentclass[../report.tex]{subfiles}
\begin{document}	
	
\chapter{State of the art}
The most interesting aspect of fabricating optical waveguides from silicon is based on low primary cost of material, the mature and evolved manufacturing techniques owing to decades of research and the potential to develop monolithic \gls{ic} using the same substrate.

	\section{Theoretical background of \gls{soi} waveguide}
Although silicon is the optimal material for electronics, only recently silicon is being considered as a practical option for \gls{oeic} solutions. Silicon has many properties conducive to fiber optics. The band gap of silicon ($\sim$\SI{1.1}{\electronvolt}) is such that the material is transparent to wavelengths commonly used for optical transport ($\sim$\SI{1.3}{\micro\metre}-\SI{1.6}{\micro\metre}). One can use standard \gls{cmos} processing techniques to sculpt optical waveguides onto the silicon surface. Similar to an optical fiber, these optical waveguides can be used to confine and direct light as it passes through the silicon \cite{reed_silicon_2004}. Due to the wavelengths typically used for optical transport and silicon’s high index of refraction, the feature sizes needed for processing these silicon waveguides are on the order of \SI{0.5}{\micro\metre}-\SI{1}{\micro\metre}. The lithography requirements needed to process waveguides with these sizes exist today. If we push forward to leading-edge research currently under way in the area of \gls{pbg} devices, today’s state-of-the-art \SI{90}{\nano\metre} fabrication facilities should meet the technical requirements needed for processing \gls{pbg} devices. What this says is that we may already have all or most of the processing technologies needed to produce silicon-based photonic devices for the next decade. In addition, the same carriers used for the basic functionality of the transistor (i.e. electrons and holes) can be used to modulate the phase of light passing through silicon waveguides and thus produce ‘active’ rather than passive photonic devices. Finally, if all this remains \gls{cmos}-compatible, it could be possible to process transistors alongside photonic devices, the combination of which could bring new levels of performance, functionality, power and size reduction, all at a lower cost.
		
		\subsection{Maxwell's equations}
To understand the........
		\subsection{Eigenvector and eigenvalue}
				
		\subsection{Optical polarization and modes}
			
			\subsubsection{TE mode}
						
			\subsubsection{TM mode}
			
		\subsection{Optical waveguides}
			
			\subsubsection{Planar waveguides}
			
			\subsubsection{Rib waveguides}			
		
		\subsection{Jones vector}
		
		\subsection{Transfer matrix (S-matrix)}
		
		\subsection{Poincaré sphere}
		
		\subsection{Polarization in optical waveguides}
		
	\section{Theoretical background of \gls{mems}}
	
		\subsection{Overview}
		
		\subsection{Actuation principle}
			
	\section{Polarization rotator}
	
		\subsection{Passive polarization rotator}
	
		\subsection{Active polarization rotator}
\end{document}
