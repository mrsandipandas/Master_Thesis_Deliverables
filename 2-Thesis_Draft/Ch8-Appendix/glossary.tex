\documentclass[../report.tex]{subfiles}
\begin{document}	
	
	\newacronym{iot}{IoT}{Internet of Things}
	\newacronym{ims}{IMS}{IP Multimedia Subsystem}
	\newacronym{ic}{IC}{Integrated circuit(s)}
	\newacronym{ict}{ICT}{Information and communication technology}
	\newacronym{cmos}{CMOS}{Complementary metal-oxide semiconductor}
	\newacronym{soi}{SOI}{Silicon on Insulator}
	\newacronym{ri}{RI}{Refractive index}
	\newacronym{mems}{MEMS}{Microelectromechanical systems}
	\newacronym{cagr}{CAGR}{Compound annual growth rate}
	\newacronym{pmd}{PMD}{polarization mode dispersion}
	\newacronym{pdl}{PDL}{polarization dependent loss}
	\newacronym{pdw}{PD$\lambda$}{polarization dependent wavelength characteristics}
	\newacronym{per}{PER}{Polarization extinction ratio}
	\newacronym{pr}{PR}{polarization rotator}
	\newacronym{tpr}{TPR}{Tunable polarization rotator}
	\newacronym{tpps}{TPPS}{Tunable polarization phase shifters}
	\newacronym{sop}{SOP}{state of polarization}
	\newacronym{oeic}{OEIC}{Optoelectronic integrated circuit}	
	\newacronym{pbg}{PBG}{Photonic band-gap}
	\newacronym{em}{EM}{Electromagnetic}
	\newacronym{tem}{TEM}{Tansverse Electromagnetic}
	\newacronym{te}{TE}{Transverse Electric}
	\newacronym{tm}{TM}{Transverse Magnetic}
	\newacronym{fem}{FEM}{Finite element method}
	\newacronym{bpm}{BPM}{Beam propagation method}
	\newacronym{fit}{FIT}{Finite integration technique}
	\newacronym{fdtd}{FDTD}{Finite difference time domain}
	\newacronym{smc}{SMC}{Single-mode condition}	
	\newacronym{rf}{RF}{Radio frequency}
	\newacronym{fom}{FOM}{figures of merit}	
\end{document}

%from documentation
%\newacronym[⟨key-val list⟩]{⟨label ⟩}{⟨abbrv ⟩}{⟨long⟩}
%above is short version of this
% \newglossaryentry{⟨label ⟩}{type=\acronymtype,
% name={⟨abbrv ⟩},
% description={⟨long⟩},
% text={⟨abbrv ⟩},
% first={⟨long⟩ (⟨abbrv ⟩)},
% plural={⟨abbrv ⟩\glspluralsuffix},
% firstplural={⟨long⟩\glspluralsuffix\space (⟨abbrv ⟩\glspluralsuffix)},
% ⟨key-val list⟩}

% If you want to link acronym and glossary: (http://tex.stackexchange.com/questions/8946/how-to-combine-acronym-and-glossary)
%\newglossaryentry{apig}{
%	name={API},
%	description={An Application Programming Interface (API) desc}}
%\newglossaryentry{api}{
%	type=\acronymtype, 
%	name={API}, 
%	description={Application Programming Interface}, 
%	first={Application Programming Interface (API)\glsadd{apig}}, 
%	see=[Glossary:]{apig}}