\documentclass[../report.tex]{subfiles}
\begin{document}
		
\chapter{Conclusions and Future work}
% Should relate to the introduction. What this works means in the real world?

\section{Conclusions}
The idea of electrically controllable optical polarization rotation, utilizing \gls{mems} in silicon photonics establishes new horizons for on-chip integrated photonics. Dynamic control of optical polarization rotation can be utilized to realize a new class of components in integrated photonics including polarization mode modulators, multiplexers, filters, and switches for advanced optical signal processing, coherent communications, and sensing. The \gls{tpr} can also be used for developing integrated polarization stabilizer, \gls{pmd} mitigation, \gls{pdl} mitigation and \gls{pmd} and \gls{pdl} measurement systems. Also, since the power consumption of the \gls{tpr} is very low, this can be used for reconfiguration of network topology at low power. Additionally, \gls{tpr} can help in multiplying the data rate of the existing deployed network infrastructure. Furthermore, the concept can be useful in situations where polarization tuning is necessary under adiabatic conditions e.g. in photon entanglement. Photon entanglement promises the development of even smaller micro-electronic devices along with secure communication channels, which can be very helpful in the future. \par 

In this thesis work, the polarization tuning capability between the two fundamental modes (\gls{te} \& \gls{tm}) has been characterized. Although, the design was simulated to achieve a \gls{tpr} in C and L bands with \gls{per} $\geq$ \SI{10}{\decibel}, but at present, a \gls{per} of \SI{5}{\decibel} - \SI{10}{\decibel} was achieved for a limited bandwidth (\SI{1530}{\nano \meter} - \SI{1570}{\nano \meter}) with a \gls{tpr} length of \SI{25}{\micro \meter}. In addition to the \gls{tpr}, in this thesis work, a free standing \gls{pbs} of length \SI{1.4}{\micro \meter} with air cladding (the shortest reported to-date to our knowledge) is designed which can be helpful in spectrometric analysis of deconfined modes. Also, in this thesis, a test setup with grating couplers, \gls{te} tapers, free-standing \gls{tm} tapers with bridges and \gls{pbs} is developed which can be used further to evaluate integrated photonic components with smaller mode sizes. All these concepts and devices will be helpful in the future where only light would be the fundamental means of communication in integrated electronics.

\section{Future work} 
Since, the \gls{mems} \gls{tpr} could not be developed with high \gls{per} due to several problems, the first step would be pattern the stair waveguide with optimized dimensions (200$\times$230 nm with the \gls{mems} cantilever at \SI{100}{\nano \meter} gap) and characterize it. Then the mode evolution design can be evaluated with a redesigned \gls{mems} cantilever (for optimized \gls{pr} cancellation). Moreover, more simulations can be carried out on the cantilever design to avoid the stiction and out-of-plane bending problem. To make the device broadband a cascade of \gls{tpr} design can be evaluated furthermore if necessary. Also, more analysis needs to be done to understand the interference pattern in the \gls{tm} transmission. Furthermore, more characterization of each of the components must be done to normalize the results of the \gls{mems} \gls{tpr}. Finally, the \gls{mems} \gls{tpr} design needs to be characterized on the Poincaré sphere to understand the \gls{sop} at different voltages. 

\end{document}
