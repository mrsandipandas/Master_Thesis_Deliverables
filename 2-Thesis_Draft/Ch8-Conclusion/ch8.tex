\documentclass[../report.tex]{subfiles}
\begin{document}
		
\chapter{Conclusions}
% Should relate to the introduction. What this works means in the real world?

The idea of electrically controllable optical polarization rotation utilizing \gls{mems} in silicon photonics establishes new horizons for on-chip integrated photonics. Dynamic control of optical polarization rotation can be utilized to realize a new class of components in integrated photonics including polarization mode modulators, multiplexers, filters, and switches for advanced optical signal processing, coherent communications, and sensing. The \gls{tpr} can also be used for developing integrated polarization stabilizer, \gls{pmd} mitigation, \gls{pdl} mitigation and \gls{pmd} and \gls{pdl} measurement systems. Also, since the power consumption of the \gls{tpr} is very low, this can be used for reconfiguration of network topology at low power. Additionally, \gls{tpr} can help in multiplying the data rate of the existing deployed network infrastructure. Furthermore, the concept can be useful in situations where polarization tuning is necessary under adiabatic conditions e.g. in photon entanglement. Photon entanglement promises the development of even smaller micro-electronic devices along with secure communication channels, which can be very helpful in the future. In this thesis work, the polarization tuning capability between the two fundamental modes has been characterized. \par

Although, the design was simulated to achieve a \gls{tpr} in C and L bands, a \gls{per} of \SI{10}{\decibel} was achieved currently for a limited spectrum. In addition to the \gls{tpr}, in this thesis work a free standing \gls{pbs} of length \SI{1.4}{\micro \meter} with air cladding (shortest till-date) is designed which can be helpful in spectrometric analysis of deconfined modes. Also, in this thesis a test setup with grating couplers, \gls{te} and \gls{tm} tapers, \gls{pbs} is developed which can be used further to evaluate integrated photonic components with smaller mode sizes. All these concepts and devices will be helpful in the future where only light would be the fundamental means of communication in integrated electronics.

\end{document}
