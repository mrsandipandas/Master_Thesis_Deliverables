\documentclass[../report.tex]{subfiles}
\begin{document}
		
\chapter{Conclusions}
% Should relate to the introduction. What this works means in the real world?

The demonstration of electrically controllable optical polarization rotation utilizing \gls{mems} in silicon photonics establishes new horizons for on-chip integrated photonics. In addition to polarization rotators, dynamic control of optical polarization rotation can be utilized to realize a new class of components in integrated photonics including polarization mode modulators, multiplexers, filters, and switches for advanced optical signal processing, coherent communications, and sensing. Advanced sensors can be designed since more spectrometric analysis can be done using tunable modes. The \gls{tpr} can also be used for developing integrated polarization stabilizer, \gls{pmd} mitigation, \gls{pdl} mitigation and \gls{pmd} and \gls{pdl} measurement systems. Also, since the power consumption of the \gls{tpr} is very low, this can be used for reconfiguration of network topology at low power. Furthermore, if necessary, a cascade of \gls{mems} \gls{tpr} can be designed to yield broadband polarization rotation without the bandwidth constraints. Lastly, the measurement setup developed can be reused further to test future integrated photonics components using TE and TM modes.

\end{document}
