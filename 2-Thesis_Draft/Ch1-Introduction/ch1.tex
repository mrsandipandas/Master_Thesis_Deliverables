\documentclass[../report.tex]{subfiles}
\begin{document}	
	
\chapter{Introduction} 
Communication and collected thinking is the key to the development and continuous evolution of human civilization which is driven by data - ”The new oil of this digital era”. With the advent of \gls{iot}, it has been estimated that by 2020 there will be 50 billion connected devices and all that can be connected will be connected. Today only 40\% of the world’s population use the Internet and the amount of data we produce per minute through different social media platform like Facebook, Youtube, Twitter, Snapchat, Instagram, Buzzfeed, Tinder, Vine, Whatsapp etc. is growing exponentially (Info-graphic example). Also, with the advent of mobile communications, there has been a huge surge in the voice as well as data traffic all over the world. It has been estimated in Ericsson’s mobility report (June, 2015) that 90\% of the world’s population over 6 years old will have a mobile phone by 2020. Currently, the telecommunications industry is moving towards IMS core networks which will mostly be IP based. So how is this humongous traffic managed? The answer is the optical fiber, which serves as the backbone of all the communication systems. The performance of optical fiber network is unprecedented and it is this backbone which gives us an unfathomable user experience. The current internet architecture has already pushed the optical fiber to the network edges and the trend is to push it as closer to the processor as possible. This has already opened up a new area of 'Siliconizing' photonics based on the decades of research obtained from microelectronics industry. The electronics industry have pushed the boundaries of the processing speed of integrated circuits after the invention of semiconductors and are currently at the edge where Moore’s Law is at its upper limit. Although, with the current technology we have a decent processing power but what about the interconnects between the ICs. Transmission of electrons through copper would definitely slow things down. Think of a data center processing petabytes of data per minute, where interconnects between processors add a significant delay. These losses can be overcome by adding optical interconnects using the current technology which can also operate at low power catering to the needs of the environment.  
	
	\section{Motivation}
The movement of data in a computer is almost the converse of movement of traffic in a city.	(http://spectrum.ieee.org/semiconductors/optoelectronics/linking-with-light). Intel processor speed and bus speed comparison shows that although we have achieved good processing speed, the interconnects are still a thing of bother. Obviously, the closer engineers can bring the optical superhighway to the microprocessor, the fewer copper bottlenecks can occur. Until recently, exponential increases in the speed, efficiency, and processing power of conventional electronic devices were achieved largely through the downscaling and clustering of components on a chip. However, this trend toward miniaturization has yielded unwanted effects in the form of significant increases in noise, power consumption, signal propagation delay and aggravates already to serious thermal management problems. Alternatively, the wires can be made fatter, but then you'll run out of space.  As a result, traditional microelectronics will soon fall short of meeting market needs, inhibited by the thermal and bandwidth bottlenecks inherent in copper wiring. Photons don't suffer from these limitations; their biggest problems are absorption and attenuation, neither of which is an issue over the distances inside a computer, or even across a room. Today Silicon Photonics Technology is a new approach to make optical devices out of silicon and use light (photons) to move huge amounts of data at very high speeds with extremely low power over a thin optical fiber rather than using electrical signals over a copper cable. Since, already enough capital investments has been done on current fabrication technology and infrastructure, engineers are working on creating monolithic design of integrated circuits which use light in place of electric signals. Organizations are trying to bridge this gap by creating highly integrated photonic and electronic components that combine the functionality of conventional CMOS circuits with the significantly enhanced system performance of photonic solutions. By allowing for the seamless integration of optical and electronic components on silicon-based substrates, this technology holds the key to fulfilling market needs for higher bandwidth and processing speed at lower power and cost.

	\section{Objectives}
The current situation landscape in Silicon Photonics is quite exciting and a number of problems are there in the industry to realize a final optimal and robust prototype. The main objective of this thesis work is to design and fabricate low power tunable polarization rotator which can primarily mitigate the effects of dispersion in fiber optic communications. To tune at sufficient low power, MEMS (Microelectromechanical Systems) technology will be used along with photonics. The waveguides will be build upon current Silicon fabrication technology which will be characterized using an automated measurement setup to minimize human errors. In this thesis following objectives will be addressed:
\begin{itemize}
	\item[$\square$] How to design a MEMS tunable polarization rotator?
	\item[$\square$]  How to simulate the design using FEM (Finite Element Modeling) techniques like Comsol, CST etc.?
	\item[$\square$]  How to effectively characterize the system with the measurements available?
	\item[$\square$]  By how much can the power consumption be reduced in comparison to current state of the art?
	\item[$\square$]  What are the different scenarios/flows to reduce the power consumption of polarization rotator?
	\item[$\square$]  How robust and efficient the solution is in terms of real deployment?
\end{itemize}

	\section{MEMS and silicon photonics}
Microelectromechanical Systems (MEMS) are micrometre-sized sensors and actuators that are used in many devices in our everyday life. Using the same fabrication technology, one can fabricate on-chip optical circuits, which drastically improve the performance of telecommunication systems. However, both MEMS and silicon waveguides have been independently developed. This project aims to bring together both fields to try out new ideas to enable new applications and to improve existing ones. 
	
	\section{Importance of these systems}
Silicon photonics is on the verge of creating a newer technology trend. It is on the brink of creating a technological revolution where semiconductor industry stood back 60 years from now just before the discovery of transistors. If this technology is developed successfully it can keep up with the tremendous data processing power needed by the current networking servers to deliver services. Since use of optics can also increase the bandwidth in a many-fold manner it can cater the needs of data rate needed by the futuristic networks. Also, as the vision is to move everything closer to the chip this technology can deliver ICs which can be easily fit into futuristic devices. Using MEMS for the tuning will ensure high resolution at very lower requirements. This is important because light transmitted to an optical circuit will typically come from an optical fibre, and optical fibres emit light with random polarization. Theses systems can expand the boundary of optical networks and bring it closer to the end devices by transgressing the network edges.
	
	\section{Outline of this thesis}
The outline of the thesis is as follows: Background, motivation and the research
questions being addressed, is discussed in Chapter 1. In Chapter 2, the current state of art for the available solution is discussed. Chapter 3 discusses about the design of the final system and the results obtained about the simulation setup. In Chapter 4, documentation about the fabricated design is provided along with currently available standard fabrication technologies. Results and characterization are an important part of the work, which is discussed in Chapter 5. Finally, Chapter 6 and 7 discusses about the conclusion and future work possibilities respectively along with the known limitations of the system if any.  
	
\end{document}
