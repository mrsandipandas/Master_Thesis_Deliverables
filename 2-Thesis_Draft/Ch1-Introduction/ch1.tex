\documentclass[../report.tex]{subfiles}
\begin{document}	

% Introduction (A way of representing - KTH Lecture)
% --------------------------------------------------
% A. Known/Background
%    1. General
%    2. Specific problem
% B. Unknown problem (Where is the hole?)
% C. Research Purpose (Question you are going to answer)
% D. Experimental approach (What the question is and how are you going to address it?)
% Explain briefly about the scope of the problem
	
\chapter{Introduction} 
	\section{Optical communication}
Communication and collective thinking are the key to the development of human civilization. This development is driven by data - “The new oil of this digital era”. With the advent of \gls{iot}, there has been a huge surge in data traffic all over the world. It has been estimated that by 2020 there will be 26 billion connected devices \cite{gartner_iot}, and all devices that can be connected will be connected. Ericsson's mobility report \cite{ericsson_mobility_report} estimates that 70\% of world's population will use smart-phones by 2020 and 90\% of the world's population over 6 years old will have a mobile phone by 2020. Today, only about 40\% \cite{internet_users} of the world’s population use the internet. With more users and different connected devices, eventually data traffic is poised to grow exponentially, as shown in Fig. \ref{fig:1_data_traffic_forecast}, as per Ericsson \cite{ericsson_traffic_exploration}.

\begin{figure}[h]
	\centering
	\includegraphics[width=1\textwidth]{1-data-traffic-forecast}
	\caption{Data traffic growth forecast by 2020, as per Ericsson, generated using \cite{ericsson_traffic_exploration}}
	\label{fig:1_data_traffic_forecast}
\end{figure}


\begin{figure}[!tbp]
	\centering
	\includegraphics[width=0.75\textwidth]{1-Internet-minute}
	\caption{What happens on internet per minute \cite{internet_minute}}
	\label{fig:1_internet_minute}
\end{figure}
Currently, as illustrated in Fig. \ref{fig:1_internet_minute}, huge data is processed per minute due to different \gls{ict} services. Eventually, as more and more people use these different \gls{ict} services on different devices, this data growth will be higher than ever. In consequence, the telecommunications industry has started moving towards \gls{ims} core networks, which is transition towards full IP based networks. So how is this traffic managed? The answer is the optical fiber, which serves as the backbone of all the communication systems. Optical fiber is chosen over previously used copper cables for the following reasons:
\begin{itemize}
	\item[$\square$] Fiber provides more bandwidth than copper and has standardized performance up to 100 Gbps and beyond, with very low power consumption.
	\item[$\square$] Fiber is less susceptible to temperature fluctuations than copper \todo{Less than copper...verify} and can be submerged in water for intercontinental long distance communication. Unlike copper, it’s immune to \gls{em} interference. 
	\item[$\square$] It doesn’t radiate signals and is extremely difficult to tap, which provides better security than copper cables.
	\item[$\square$] Fiber optic transmission results in less attenuation (losses) than copper cables.
\end{itemize}

	\section{Silicon photonics}
The performance of optical fiber networks is remarkable and it is this backbone which gives us a great user experience. The current internet architecture has already pushed the optical fiber to the network edges and the trend is to push it as closer to the processor as possible. This has already opened up a new trend of “siliconizing photonics” \cite{silicon_photonics}, which arose from the decades of research from microelectronics industry.\par 
The electronics industry has pushed the boundaries of the processing speed of \gls{ic} according to Moore’s Law. Until recently, exponential increases in the speed, efficiency, and processing power of conventional electronic devices were achieved largely through the downscaling and clustering of components on a chip. However, this trend toward miniaturization has yielded unwanted effects in the form of significant increases in noise, power consumption, signal propagation delay and aggravates already to serious thermal management problems. Alternatively, the wires can be made thicker, but then the packing density will be inefficient. As a result, traditional microelectronics will soon fall short of meeting market needs, inhibited by the thermal and bandwidth bottlenecks inherent in copper wiring. Intel processor speed and bus speed comparison shows that although we have achieved good processing speed, the interconnects always find difficulty in catching up with the processing speed \cite{intel_proc_compare}. Annual global data center IP traffic will reach 10.4 zettabytes (863 exabytes per month) by the end of 2019, up from 3.4 zettabytes per year (287 exabytes per month) in 2014 \cite{cisco_forecast_2019}. Think of a server rack in a data center processing an average of this huge data per second, where interconnects between multiple processors in the server rack add up to a significant bottleneck. These bottlenecks can be overcome by substituting copper with optical interconnects using the current technology, which can also operate at low power and better efficiency. In addition optical interconnects can also reduce heat dissipation, switching and transmission of electrical signals.\par

Although silicon is the material of choice for electronics, only from late 1980s silicon is being considered as a practical option for \gls{oeic} solutions. Silicon has many properties that make it a good material for optics. First of all, the band gap of silicon ($\sim$\SI{1.1}{\electronvolt}) is such that the material is transparent to wavelengths commonly used for optical communication ($\sim$\SI{1.3}{\micro\metre}-\SI{1.6}{\micro\metre}). Moreover, one can use standard \gls{cmos} processing techniques to sculpt optical waveguides onto the silicon surface. Similar to an optical fiber, these waveguides can be used to confine and direct light as it passes through the silicon \cite{reed_silicon_2004} using total internal reflection. Due to the wavelengths typically used for optical transport and silicon’s high index of refraction, the feature sizes needed for processing these silicon waveguides are on the order of \SI{0.5}{\micro\metre}-\SI{1}{\micro\metre}. This makes silicon excellent for miniaturization of optical components. The fabrication and lithography requirements needed to process waveguides with these sizes exist today. Finally, it is \gls{cmos}-compatible, making it possible to process monolithic optical devices, which could bring new levels of performance, functionality, power and size reduction, all at a lower cost. \par

Today silicon photonics is a new approach to make miniaturized optical devices that use light to move huge amounts of data at very high speeds with extremely low power over a thin optical fiber rather than using electrical signals over a copper wire. Since already a large capital investment has been done on perfecting the current fabrication technology and infrastructure, engineers are working on creating monolithic design of integrated circuits which will use light instead of electric signals \cite{optical_linking}. Research institutes and industry, are trying to bridge this gap by creating highly integrated photonic and electronic components that combine the functionality of conventional \gls{cmos} circuits with the significantly enhanced performance of photonic solutions. Various kinds of silicon photonic devices, such as switches \cite{stabile_integrated_2016,wu_mems-enabled_2015,nikolova_scaling_2015,lu_low-power_2014}, modulators \cite{dong_silicon_2015,chen_generation_2013}, photo-detectors \cite{urino_demonstration_2012,chang_high-power_2015}, delay lines \cite{garcia_design_2015,mattarei_variable_2014}, sensors \cite{janz_silicon_2007,lim_laser_2010,ryckeboer_glucose_2014} etc. have been reported till date. This leads to a booming silicon photonics market, which is estimated to grow to 700 million USD by 2024 \cite{jalali_silicon_2006,silicon_photonics_growth_2015} with a \gls{cagr} of 38\%. 

	\section{Motivation} 
To keep up with bandwidth requirements using existing network infrastructure, spatial-division multiplexing technique \cite{space_richardson_2013} is being contemplated, which uses multimode transmission. However, simply connecting the end of such fibers to an \gls{oeic} is far more complicated than standard fibers, because much more mechanical precision is required. Great care has to be taken to make sure light goes in exactly as intended \cite{hecht_is_2016}. Moreover, all photonic devices based on silicon waveguides are sensitive to polarization due to large structural birefringence, which induces substantial \gls{pdl}, \gls{pmd}, and other \gls{pdw}, limiting their usability. Also, in a complex \gls{oeic} system, polarization is a major issue because power can be exchanged between the polarization states in the presence of junctions, tapers, slanted sidewalls, bends, or other discontinuities. Therefore, sometimes, it is necessary to have a fixed degree of polarization state, and it may also be necessary to rotate an incoming polarization state. \par

To overcome these challenges, \gls{pr} is engineered on silicon for \gls{oeic} and various designs have already been demonstrated \cite{xie_efficient_2015,velasco_ultracompact_2012,leung_numerical_2011,wang_design_2014,dai_novel_2011,wirth_efficient_2012,chen_compact_2011}. The main working principle of these proposed solutions is introducing asymmetry in the waveguide structure. However, in some cases, the designs \cite{sarmiento-merenguel_demonstration_2015}, if used in commercial applications for miniature interconnects, would incur inefficient packing density since too much space is required to achieve high and robust tuning. Apart from that there might also be thermo-optic cross-talk problem which might change phase of the wave in other waveguides in high compact density environment as silicon is highly susceptible to thermal changes \cite{ibrahim_athermal_2012}. Also, a \gls{tpr} has been reported which works on the principle of Berry’s phase, a quantum-mechanical phenomenon of purely topological origin \cite{xu_electrically_2014}. For \gls{pr}, the design uses out-of-plane ring cavity which inherits the narrow band spectral features of ring resonator limiting the bandwidth. Phase tuning is achieved through thermo-optic effect which has its limitations. The general goal of the thesis work is to realize an efficient \gls{tpr} using \gls{mems} in C and L bands, at low power, with high precision and accuracy.    
	\section{Objectives}
\textbf{Main objective}: To design and fabricate low power \gls{tpr} based on \gls{mems} tuning. \\

\noindent \textbf{Sub objectives}: The areas which will be addressed are:
\begin{itemize}
	\item[$\square$] Feasibility of the idea and the strategy to design the \gls{tpr}. 
	\item[$\square$] Design and simulation of the device.
	\item[$\square$] Fabrication and characterization of the MEMS \gls{tpr}.
\end{itemize}
	
	\section{Outline of this thesis}
The outline of the thesis is as follows: Background, motivation and the research questions being addressed, is discussed in Chapter 1. In Chapter 2, the current state of art for the available solution is discussed along with the background literature required. Here, also the working principle of the current available design are explained along with the areas which can be improved. Chapter 3 discusses about the design of the final system and the results obtained about the simulation setup. In Chapter 4, documentation about the fabricated design is provided along with currently available standard fabrication technologies. Results and characterization are an important part of the work, which is discussed in Chapter 5. Finally, Chapter 6 and 7 discusses about the future work possibilities respectively along with the known limitations and conclusion of the system if any.  
	
\end{document}
