\documentclass[../report.tex]{subfiles}
\begin{document}	
	\chapter{Discussion}
\gls{mems} \gls{tpr} can establish new horizons for on-chip integrated photonics. The power consumption of the \gls{mems} \gls{tpr} is very low and thus can be used for low powered devices. Furthermore, if the current problems are fixed, a cascade of \gls{mems} \gls{tpr} can be designed to yield broadband polarization rotation without the bandwidth constraints. The measurement setup developed during the thesis work with grating couplers, tapers and \gls{pbs} can be reused further to test future integrated photonic components which uses \gls{te} and \gls{tm} modes. Also, a shortest free-standing \gls{pbs} with air cladding has been designed during this thesis, which can be useful for deconfined mode coupling with high packing density.
	
\begin {table}[H]
\begin{center} 
	\begin{tabular}{ | m{8em} | m{1.7cm}| m{1.7cm} | m{2.8cm} | m{10em} | }  
		\hline
		\textbf{TPR designing principle} & \textbf{Length} & \textbf{PER} & \textbf{Bandwidth} & \textbf{Limitations} \\ [1.5ex]
		\hline\hline
		\gls{tpr} using phase shifter [\ref{concept:tpps}] & $\sim$ \SI{700}{\micro \meter} & \SI{40}{\decibel} & 1530 - \SI{1570}{\nano \meter} & Thermal cross-talk, Inefficient packing density \\ 
		\hline	
		\gls{tpr} using Berry's phase [\ref{concept:berry_phase}] & $>$ \SI{200}{\micro \meter} & \SI{19}{\decibel} & 1556 - \SI{1562}{\nano \meter}  &  Narrow bandwidth, Thermal cross-talk, Inefficient packing density \\ 
		\hline
		\gls{mems} \gls{tpr} & $\sim$ \SI{25}{\micro \meter} & 5 - \SI{10}{\decibel} & 1530 - \SI{1570}{\nano \meter} & Low PER in the current fabricated product, \gls{mems} cantilever stiction issue \\ [1ex] 
		\hline
	\end{tabular}
\end{center}
\caption {Comparative analysis of the \gls{mems} \gls{tpr} with the state-of-the-art \gls{tpr}} \label{table:tpr_comparision} 
\end {table}

The comparative analysis of the \gls{mems} \gls{tpr} designed in the thesis shows a significant improvement in terms of device length. Also, as the tuning is done mechanically, it is devoid of thermal cross-talk problem.	Although, the \gls{mems} \gls{tpr} has some problems, it can be improved to achieve better results in terms of \gls{per} and bandwidth.

	\section{Limitations}
The development process of the \gls{tpr} gave a broad understanding of the working and fabrication of micro-optic based devices. The outcome of the understanding are discussed as follows:
\begin{itemize}	
	\item[$\square$] \textbf{Mode evolution design issue:} In the thesis work the mode hybridization based design is used for the \gls{tpr}. However, in the literature review and simulation work performed, mode evolution based design showed promising results. But this design was not used since the length of the device is difficult to estimate. This estimation would have taken more fabrication iterations. 
	
	\item[$\square$] \textbf{200$\times$230 nm dimension stair waveguide, with a cantilever at 100 nm gap patterning problem:} In the simulation, the best dimensions of the stair waveguide were obtained at 200$\times$\SI{230}{\nano\meter}. However, it is difficult to pattern such dimensions due to proximity effect. Hence, it was only possible to pattern a device dimension of 160$\times$\SI{260}{\nano\meter}. Moreover, the gap between the \gls{mems} waveguide and the passive \gls{pr} waveguide was \SI{100}{\nano\meter} during simulation. However due lateral stiction problem the \gls{mems} waveguide and the passive \gls{pr} were stuck together. Hence, a gap of \SI{200}{\nano\meter} was used.
	
	\item[$\square$] \textbf{Thinner resist:} The resist used during the fabrication procedure must be thinner as thicker resist on the small device layer can break the device pattern.
	%\item[$\square$] \textbf{High selectivity of resist:}
	
	\item[$\square$] \textbf{Small write-field alignment:} Write-filed alignment is a very central adjustment in the process of getting the best possible e-beam lithographic result. Smaller write-field alignment ensures precise positioning of the stage under the e-beam deflection system.
	
	\item[$\square$] \textbf{Cantilever stiction problem:} If the \gls{mems} cantilever are not characterized properly, they can stick to the bottom and can be damaged permanently.
	
	\item[$\square$] \textbf{Cantilever bending problem:} Longer cantilever bends in the fabrication process. Hence,  it is necessary to design shorter cantilevers. Shorter cantilevers are stiffer and do not bend out of plane.
	
	\item[$\square$] \textbf{Dry-etching problem:} The gratings and the stair waveguide are designed for \SI{110}{\nano \meter} slab height. However, the dry-etch procedure in the lab is difficult to control to achieve precise etching. So, device tolerance simulations should be performed to analyze the results.  
	%\item[$\square$] \textbf{Not too fast as using MEMS:}
	%\item[$\square$] \textbf{Narrowband / Broadband?:}
	
\end{itemize}
	
	\section{Future work} 
	Since, the \gls{mems} \gls{tpr} could not be developed with high \gls{per} due to several problems, the first step would be pattern the stair waveguide with optimized dimensions (200$\times$230 nm with the \gls{mems} cantilever at 100 nm gap) and characterize it. Then the mode evolution design can be evaluated with a redesigned \gls{mems} cantilever (for optimized \gls{pr} cancellation) and tried out. Moreover, more simulations can be carried out on the cantilever design to avoid the stiction and out-of-plane bending problem. To make the device broadband a cascade of \gls{tpr} design can be evaluated furthermore if necessary. 
	
\end{document}
