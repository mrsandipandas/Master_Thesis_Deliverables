\documentclass[../main.tex]{subfiles}
\begin{document}
	\chapter*{\centerline{Abstract}}

	There has been a huge surge in data traffic all over the world due to the rise of streaming and media services as well as connected devices. The current Internet architecture has already pushed the optical fiber to the network edges and the trend is to push it as close to the processor as possible, which has opened up a new trend of “siliconizing photonics”. Silicon photonics is helpful for making miniaturized optical devices that use light to move huge amounts of data at very high speeds with extremely low power. \\
	
	Polarization rotation can be helpful for spatial-division multiplexing techniques, which will boost the network bandwidth by multi-mode transmission using the existing deployed network infrastructure. However, great mechanical precision and substantial polarization tuning is required for multi-mode transmission, which is solved by the tunable polarization rotator. The dynamic control of polarization rotation can also be utilized to realize a new class of components in integrated photonics including polarization mode modulators, multiplexers, filters, as well as switches for advanced optical signal processing, coherent communications, and sensing. \\
	
	This thesis work introduces a novel way of tuning optical polarization by using micro electro-mechanical systems (MEMS). When voltage is applied to a MEMS tunable cantilever which holds a secondary waveguide, a mechanical movement occurs. As a result, the evanescent field of the resonant mode in the primary waveguide is perturbed from the side because of the change in effective index of refraction of the mode. Tuning by electrostatic actuation can potentially provide a broad wavelength tuning unlike the current thermal polarization tuning solutions available. The overall, packing density is also high due to low thermal cross-talk. Currently, a MEMS tunable polarization rotator of length \SI{25}{\micro \meter} is designed in this thesis work, with a polarization extinction ratio of \SI{10}{\decibel}, which works in \SI{1530}{\nano \meter} - \SI{1570}{\nano \meter} wavelength spectrum. \\
	
	In addition to the MEMS tunable polarization rotator, in this thesis work, a free standing polarization beam splitter of length \SI{1.4}{\micro \meter} with air cladding (shortest till-date to my knowledge) is designed which can be helpful in spectrometric analysis of deconfined modes. Also, during this thesis work, a test setup with grating couplers, TE tapers, free-standing TM tapers with bridges and polarization beam splitter is structured which can be used further to evaluate integrated photonic components with smaller mode sizes.
\end{document}
