\documentclass[../main.tex]{subfiles}
\begin{document}
	\chapter*{\centerline{Abstract}}

	There has been a huge surge in data traffic all over the world due to the rise of streaming media services and connected devices. The current internet architecture has already pushed the optical fiber to the network edges and the trend is to push it as close as possible, to the processor. Silicon photonics addresses this challenge by enabling miniaturized optical devices that use light to move huge amounts of data at very high speeds with extremely low power. To further improve the data transmission capacity, one can make use of different polarizations of light. However, to take advantage of different polarizations, devices with on-chip polarization rotation capability are required. This is achieved by a tunable polarization rotator. Moreover, full control of polarization rotation can also be utilized to realize a new class of components in integrated photonics including polarization mode modulators, multiplexers, filters, as well as switches for advanced optical signal processing, coherent communications, and sensing. \\

	This thesis work introduces a novel tunable polarization rotator that uses microelectromechanical systems (MEMS) as its actuation principle. When voltage is applied to a MEMS tunable cantilever which holds a silicon beam, a mechanical movement occurs, which in turn affects the optical mode shape travelling through a waveguide. In this work, we have designed, fabricated, and presented preliminary results for a MEMS tunable polarization rotator with a polarization extinction ratio of \SI{10}{\decibel}, which works in \SI{1530}{\nano \meter} - \SI{1570}{\nano \meter} wavelength spectrum. In addition to the MEMS tunable polarization rotator, in this thesis work, a free standing polarization beam splitter of length \SI{1.4}{\micro \meter}, the shortest to-date to our knowledge, was designed, fabricated, and characterized. Polarization beam splitter is helpful in designing switches and multiplexer/de-multiplexer for silicon photonics based components. Since, the beam splitter design is free standing, it can also be easily used for sensing of evanescent fields. 
\end{document}
