\documentclass[../report.tex]{subfiles}
\begin{document}	
	
\chapter{Fabrication}

\section{Process}
The \gls{mems} tunable device was fabricated using a SOI-based process with two dry etch steps for the silicon device layer (resulting in two heights) and a wet $SiO_2$ under-etch \cite{errando-herranz_low-power_2015}. The first lithography step defines the ridge waveguides that form the ring resonator and the grating coupled bus waveguide. \todo{Update with my process thing} The second lithography step and the wet under-etch define the free standing cantilever. The cantilever is delimited by the fully etched slot waveguides, and its free suspended area is determined by the placement of etch holes.

\par The fabrication process starts with a clean SOI chip with a \SI{220}{\nano \meter} crystalline silicon device layer and \SI{2}{\micro \meter} buried oxide (\ref{fig:4_fabrication} A). This is a standard substrate specification used by the Epixfab silicon photonics foundries. Electron beam patterning of a \SI{50}{\nano \meter} layer of a high-resolution negative electron beam resist (\gls{hsq}) defines the waveguide structures (\ref{fig:4_fabrication} B). The pattern is then transferred to the device layer by timed dry etch of silicon, resulting in ridge waveguide structures with \SI{110}{\nano \meter} height on a \SI{110}{\nano \meter} thick silicon slab ((\ref{fig:4_fabrication} C)). The patterned \gls{hsq} remains on the chip for the next lithography step.

\begin{figure}[H] %h
	\centering
	\includegraphics[width=0.5\textwidth]{4-fabrication}
	\caption{Fabrication process}
	\label{fig:4_fabrication}
\end{figure} \todo{Update E,F,G}

\section{Final Product}

Show SEM image

\end{document}
